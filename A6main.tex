%%%%%%%%%%%%%%%%%%%%%%%%%%%%%%%%%%%%%%%%%%%%%%%%%%%%%%%%%%%%%%%
%
% Welcome to Overleaf --- just edit your LaTeX on the left,
% and we'll compile it for you on the right. If you open the
% 'Share' menu, you can invite other users to edit at the same
% time. See www.overleaf.com/learn for more info. Enjoy!
%
%%%%%%%%%%%%%%%%%%%%%%%%%%%%%%%%%%%%%%%%%%%%%%%%%%%%%%%%%%%%%%%


% Inbuilt themes in beamer
\documentclass{beamer}

% Theme choice:
\usetheme{CambridgeUS}

% Title page details: 
\title{ASSIGNMENT-6 \\ PAPOULLIS PROBLEM}  
\author{MARRI SATHVIKA \\ AI21BTECH11020}
\date{30th May,2022}

\begin{document}

% Title page frame
\begin{frame}
    \titlepage 
\end{frame}

\begin{frame}{QUESTION}
Show that if R is the correlation matrix of the random vector X : [x_1......,x_n] $and$  $R^{-1}$ is its inverse, then E[X R^{-1}X^t] = n
\end{frame}

\begin{frame}{SOLUTION}
  If R^{-1} = \begin{bmatrix}  a_{11}.....a_{1n} \\ a_{n1}.....a_{nn}  \end{bmatrix} then \sum_{j} a_{ij} R_{ji} = 1 \\\\   
  Hence, E[XR^{-1}X^t] = E [\sum\limits_{n}^{i=1} \sum\limits_{n}^{j=1} x_{i} a_{ij} x_{j}]\\\\
    =\sum\limits_{n}^{i=1} \sum\limits_{n}^{j=1} a_{ij} R_{ji}\\ 
    = \sum\limits_{i=1}^{n} 1 = n\\\\

 \boxed{\therefore E[X R^{-1}X^t] = n}

\end{frame}

\end{document}